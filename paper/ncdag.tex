\documentclass[a4paper]{article}

%\renewcommand*{\sectfont}{\normalfont\bfseries}

\usepackage{fixltx2e}
\usepackage{isomath}

\usepackage[utf8]{inputenc}
\usepackage[T1]{fontenc}
\usepackage[british]{babel}

\usepackage{natbib}
\usepackage{proof}
\usepackage{tikz}

\usepackage{microtype}

\newcommand*{\INFER}[3][]{\ensuremath{\infer[#1]{\mathstrut #2}{\mathstrut #3}}}

\newcommand*{\GRAPHR}[2]{%
	\begin{tikzpicture}[xscale=0.5, yscale=1.5]
		\path[draw, white] (0, 0) rectangle (1, 0.5);
		\path[thick, ->=stealth] (0, 0) edge[out=90, in=90] (1, 0);
		\node[anchor=north] at (0, 0) {$\mathstrut #1$};
		\node[anchor=north] at (1, 0) {$\mathstrut #2$};
	\end{tikzpicture}%
}

\newcommand*{\GRAPHL}[2]{%
	\begin{tikzpicture}[xscale=0.5, yscale=1.5]
		\path[draw, white] (0, 0) rectangle (1, 0.5);
		\path[thick, <-=stealth] (0, 0) edge[out=90, in=90] (1, 0);
		\node[anchor=north] at (0, 0) {$\mathstrut #1$};
		\node[anchor=north] at (1, 0) {$\mathstrut #2$};
	\end{tikzpicture}%
}

\newcommand*{\GRAPHH}[2]{%
	\begin{tikzpicture}[xscale=0.5, yscale=1.5]
		\path[draw, white] (0, 0) rectangle (1, 0.5);
		\path[draw, thick, gray] (0, 0) rectangle (1, 0.25);
		\node[anchor=north] at (0, 0) {$\mathstrut #1$};
		\node[anchor=north] at (1, 0) {$\mathstrut #2$};
	\end{tikzpicture}%
}

\newcommand*{\SEQR}[2]{%
	\begin{tikzpicture}[xscale=0.5, yscale=1.5]
		\path[draw, white] (0, 0) rectangle (3.5, 0.5);
		\path[thick, ->=stealth] (0, 0) edge[out=90, in=90] (1, 0);
		\node[anchor=south] at (1.75, 0) {$\mathstrut\cdots$};
		\path[thick, ->=stealth] (2.5, 0) edge[out=90, in=90] (3.5, 0);
		\node[anchor=north] at (0, 0) {$\mathstrut #1$};
		\node[anchor=north] at (3.5, 0) {$\mathstrut #2$};
	\end{tikzpicture}%
}

\newcommand*{\SEQL}[2]{%
	\begin{tikzpicture}[xscale=0.5, yscale=1.5]
		\path[draw, white] (0, 0) rectangle (3.5, 0.5);
		\path[thick, <-=stealth] (0, 0) edge[out=90, in=90] (1, 0);
		\node[anchor=south] at (1.75, 0) {$\mathstrut\cdots$};
		\path[thick, <-=stealth] (2.5, 0) edge[out=90, in=90] (3.5, 0);
		\node[anchor=north] at (0, 0) {$\mathstrut #1$};
		\node[anchor=north] at (3.5, 0) {$\mathstrut #2$};
	\end{tikzpicture}%
}

\newcommand*{\SEQM}[2]{%
	\begin{tikzpicture}[xscale=0.5, yscale=1.5]
		\path[draw, white] (0, 0) rectangle (3.5, 0.5);
		\path[thick] (0, 0) edge[out=90, in=90] (1, 0);
		\node[anchor=south] at (1.75, 0) {$\mathstrut\cdots$};
		\path[thick] (2.5, 0) edge[out=90, in=90] (3.5, 0);
		\node[anchor=north] at (0, 0) {$\mathstrut #1$};
		\node[anchor=north] at (3.5, 0) {$\mathstrut #2$};
	\end{tikzpicture}%
}

\newcommand*{\SEQU}[2]{%
	\begin{tikzpicture}[xscale=0.5, yscale=1.5]
		\path[draw, white] (0, 0) rectangle (3.5, 0.5);
		\path[draw, thick, gray] (0, 0) rectangle (1, 0.25);
		\node[anchor=south] at (1.75, 0) {$\mathstrut\cdots$};
		\path[draw, thick, gray] (2.5, 0) rectangle (3.5, 0.25);
		\node[anchor=north] at (0, 0) {$\mathstrut #1$};
		\node[anchor=north] at (3.5, 0) {$\mathstrut #2$};
	\end{tikzpicture}%
}

\begin{document}

\title{Parsing to Noncrossing Acyclic Digraphs}

\author{%
	Marco Kuhlmann\\
	Dept.\ of Computer and Information Science\\
	Linköping University, Sweden}

\maketitle


\section{Introduction}

A \emph{noncrossing graph} of size $n$ is a graph drawn on $n$ points laid out on a circle and numbered in counter-clockwise order from 1 to $n$ such that the edges are straight-line segments that do not cross.
This note is about \emph{noncrossing acyclic digraphs}.
Their number is given by the following sequence \citep{tirrell2014number}:
\begin{center}
	1, 3, 25, 335, 5521, 101551, 1998753, 41188543, 877423873, 19166868607, \dots
\end{center}

We are interested in an algorithm that takes a number $n$ and constructs a compact representation of the set of all noncrossing acyclic digraphs of size~$n$.
More formally, 


\section{Algebraic Structure}

\begin{displaymath}
	\GRAPHR{1}{2}
	\qquad
	\GRAPHL{1}{2}
	\qquad
	\GRAPHH{1}{2}
\end{displaymath}

We build noncrossing acylic digraphs using three operations:
\begin{itemize}
	\item Concatenate two graphs, identifying the rightmost node of the first graph with the leftmost node of the second graph.
	\item Extend a given graph by adding an edge from the leftmost node to the rightmost node.
	\item Extend a given graph by adding an edge from the rightmost node to the leftmost node.
\end{itemize}

For example, there are 3 noncrossing acyclic digraphs on 2 of size 2: the graph with the edge $1 \to 2$, the graph with the edge $2 \to 1$, and the graph with no edges.

\section{Inference System}

We specify our algorithm in terms of a deduction system \citep{shieber1995principles}.
We assume that we are given a number $n \geq 2$.
The \emph{items} of our logic take one of six possible forms.
\begin{displaymath}
	\GRAPHR{i}{j}
	\qquad
	\GRAPHL{i}{j}
	\qquad
	\GRAPHH{i}{j}
\end{displaymath}
These items represent a noncrossing acyclic digraph whose leftmost node is $i$, whose rightmost node is $j$, and that has an edge $i \to j$ (first type) or $j \to i$.

\paragraph{Sequences}

The next three item types represent 

\begin{displaymath}
	\SEQR{i}{j}
	\qquad
	\SEQL{i}{j}
	\qquad
	\SEQM{i}{j}
\end{displaymath}

\paragraph{Unconnected}

\begin{displaymath}
	\SEQU{i}{j}
\end{displaymath}

The first set of rules create connected sequences.
\begin{displaymath}
	\INFER{\SEQR{i}{k}}{\GRAPHR{i}{j} & \GRAPHR{j}{k}}
	\qquad
	\INFER{\SEQL{i}{k}}{\GRAPHL{i}{j} & \GRAPHL{j}{k}}
	\qquad
	\INFER{\SEQM{i}{k}}{\GRAPHR{i}{j} & \GRAPHL{j}{k}}
	\qquad
	\INFER{\SEQM{i}{k}}{\GRAPHL{i}{j} & \GRAPHR{j}{k}}
\end{displaymath}

The following rules create unconnected sequences.
\begin{displaymath}
	\INFER{\SEQU{i}{k}}{\GRAPHR{i}{j} & \GRAPHH{j}{k}}
	\qquad
	\INFER{\SEQU{i}{k}}{\GRAPHH{i}{j} & \GRAPHR{j}{k}}
	\qquad
	\INFER{\SEQU{i}{k}}{\GRAPHL{i}{j} & \GRAPHH{j}{k}}
	\qquad
	\INFER{\SEQU{i}{k}}{\GRAPHH{i}{j} & \GRAPHL{j}{k}}
\end{displaymath}

\paragraph{Extend Sequences}

\begin{displaymath}
	\INFER{\SEQR{i}{k}}{\SEQR{i}{j} & \GRAPHR{j}{k}}
	\qquad
	\INFER{\SEQM{i}{k}}{\SEQR{i}{j} & \GRAPHL{j}{k}}
	\qquad
	\INFER{\SEQU{i}{k}}{\SEQR{i}{j} & \GRAPHH{j}{k}}
\end{displaymath}

\begin{displaymath}
	\INFER{\SEQM{i}{k}}{\SEQL{i}{j} & \GRAPHR{j}{k}}
	\qquad
	\INFER{\SEQL{i}{k}}{\SEQL{i}{j} & \GRAPHL{j}{k}}
	\qquad
	\INFER{\SEQU{i}{k}}{\SEQL{i}{j} & \GRAPHH{j}{k}}
\end{displaymath}

\begin{displaymath}
	\INFER{\SEQM{i}{k}}{\SEQM{i}{j} & \GRAPHR{j}{k}}
	\qquad
	\INFER{\SEQM{i}{k}}{\SEQM{i}{j} & \GRAPHL{j}{k}}
	\qquad
	\INFER{\SEQU{i}{k}}{\SEQM{i}{j} & \GRAPHH{j}{k}}
\end{displaymath}

\begin{displaymath}
	\INFER{\SEQU{i}{k}}{\SEQU{i}{j} & \GRAPHR{j}{k}}
	\qquad
	\INFER{\SEQU{i}{k}}{\SEQU{i}{j} & \GRAPHL{j}{k}}
	\qquad
	\INFER{\SEQU{i}{k}}{\SEQU{i}{j} & \GRAPHH{j}{k}}
\end{displaymath}

\paragraph{Cover Sequences}

\begin{displaymath}
	\INFER{\GRAPHR{i}{j}}{\SEQR{i}{j}}
	\qquad
	\INFER{\GRAPHR{i}{j}}{\SEQM{i}{j}}
	\qquad
	\INFER{\GRAPHR{i}{j}}{\SEQU{i}{j}}
\end{displaymath}

\begin{displaymath}
	\INFER{\GRAPHL{i}{j}}{\SEQL{i}{j}}
	\qquad
	\INFER{\GRAPHL{i}{j}}{\SEQM{i}{j}}
	\qquad
	\INFER{\GRAPHL{i}{j}}{\SEQU{i}{j}}
\end{displaymath}


\section{Code}

The inference system can be implemented using a bottom-up dynamic programming algorithm.
The code is available on GitHub.


\section{Conclusion}

can be used to compute highest-scoring graph under an arc-factored scoring model, count the number of graphs, compute inside and outside scores


\bibliographystyle{plainnat}
\bibliography{mcqm}

\end{document}
